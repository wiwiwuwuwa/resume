%-------------------------
% Resume in Latex
% Author : Vladimir Nesterov
% Based off of: https://github.com/sb2nov/resume
% License : MIT
%------------------------

\documentclass[a4paper,11pt]{article}

\usepackage{latexsym}
\usepackage[empty]{fullpage}
\usepackage{titlesec}
\usepackage{marvosym}
\usepackage[usenames,dvipsnames]{color}
\usepackage{verbatim}
\usepackage{enumitem}
\usepackage[hidelinks]{hyperref}
\usepackage{fancyhdr}
\usepackage[english]{babel}
\usepackage{tabularx}
\usepackage{CJKutf8}
\input{glyphtounicode}

%----------FONT OPTIONS----------
% sans-serif
% \usepackage[sfdefault]{FiraSans}
% \usepackage[sfdefault]{roboto}
% \usepackage[sfdefault]{noto-sans}
% \usepackage[default]{sourcesanspro}

% serif
% \usepackage{CormorantGaramond}
% \usepackage{charter}

\pagestyle{fancy}
\fancyhf{} % clear all header and footer fields
\fancyfoot{}
\renewcommand{\headrulewidth}{0pt}
\renewcommand{\footrulewidth}{0pt}

\urlstyle{same}

\raggedbottom
\raggedright
\setlength{\tabcolsep}{0in}

% Sections formatting
\titleformat{\section}{
  \vspace{-4pt}\scshape\raggedright\large
}{}{0em}{}[\color{black}\titlerule \vspace{-5pt}]

% Ensure that generate pdf is machine readable/ATS parsable
\pdfgentounicode=1

%-------------------------
% Custom commands
\newcommand{\resumeItem}[1]{
  \item\small{
    {#1 \vspace{-2pt}}
  }
}

\newcommand{\resumeSubheading}[4]{
  \vspace{-2pt}\item
    \begin{tabular*}{0.97\textwidth}[t]{l@{\extracolsep{\fill}}r}
      \textbf{#1} & #2 \\
      \textit{\small#3} & \textit{\small #4} \\
    \end{tabular*}\vspace{-7pt}
}

\newcommand{\resumeSubSubheading}[2]{
    \item
    \begin{tabular*}{0.97\textwidth}{l@{\extracolsep{\fill}}r}
      \textit{\small#1} & \textit{\small #2} \\
    \end{tabular*}\vspace{-7pt}
}

\newcommand{\resumeProjectHeading}[2]{
    \item
    \begin{tabular*}{0.97\textwidth}{l@{\extracolsep{\fill}}r}
      \small#1 & #2 \\
    \end{tabular*}\vspace{-7pt}
}

\newcommand{\resumeSubItem}[1]{\resumeItem{#1}\vspace{-4pt}}

\renewcommand\labelitemii{$\vcenter{\hbox{\tiny$\bullet$}}$}

\newcommand{\resumeSubHeadingListStart}{\begin{itemize}[leftmargin=0.15in, label={}]}
\newcommand{\resumeSubHeadingListEnd}{\end{itemize}}
\newcommand{\resumeItemListStart}{\begin{itemize}}
\newcommand{\resumeItemListEnd}{\end{itemize}\vspace{-5pt}}

%-------------------------------------------
%%%%%%  RESUME STARTS HERE  %%%%%%%%%%%%%%%%%%%%%%%%%%%%

\begin{document}
\begin{CJK*}{UTF8}{min}

%----------HEADING----------
\begin{center}
    \textbf{\Huge \scshape ウラジミール・ネステロフ} \\ \vspace{1pt}
    \small 日本、東京 \\ \vspace{1pt}
    +81 70 9110 8139 $|$ \href{mailto:wiwiwuwuwa@gmail.com}{wiwiwuwuwa@gmail.com} $|$ \href{https://linkedin.com/in/wiwiwuwuwa}{linkedin.com/in/wiwiwuwuwa} \\ \vspace{1pt}
    \href{https://github.com/wiwiwuwuwa}{github.com/wiwiwuwuwa} $|$ \href{https://www.youtube.com/@wiwiwuwuwa}{youtube.com/@wiwiwuwuwa} $|$ \href{https://artstation.com/wiwiwuwuwa}{artstation.com/wiwiwuwuwa}
\end{center}

%-----------SUMMARY-----------
\section{概要}
  \resumeSubHeadingListStart
    \item {
      AAAゲーム開発で5年の経験を持つC++ソフトウェアエンジニア。C++、C\#、TypeScript、およびシェーダー言語(CG/HLSL、ShaderLab)に精通。Unreal Engine、Unity3D、グラフィックスプログラミングに熟練しており、コンピュータサイエンスとゲームメカニクスの実装に強い背景を持つ。
    }
  \resumeSubHeadingListEnd

%-----------EXPERIENCE-----------
\section{経験}
  \resumeSubHeadingListStart

    \resumeSubheading
      {Injustice 2 Mobile, C++ ソフトウェアエンジニア}{2019年5月 -- 2023年12月}
      {Sperasoft (Warner Bros' NetherRealm Studioと共同開発)}{}
      \resumeItemListStart
        \resumeItem{ボスとの対戦やクラン戦を可能にする非同期マルチプレイヤー機能を含むクランシステムなど、主要なゲームメカニクスの開発と強化。ゲームパスやルートボックスなどのマネタイズ機能を作成し、ユーザーエンゲージメントを向上。}
        \resumeItem{C++ゲームクライアントとTypeScript REST APIゲームサーバー間のシームレスな相互作用を設計し、通信とパフォーマンスを最適化。}
        \resumeItem{デザイナーやアーティストと密に協力し、新しいゲームおよびインターフェース要素を設計および展開し、視覚的およびインタラクティブな面を向上。}
        \resumeItem{ゲームシェーダーと低レベルのSlateフレームワークに特化。}
        \resumeItem{JavaとObjective-Cを使用してネイティブAndroidおよびiOSコードを開発および統合し、新機能を導入し、多様なユーザーグループ向けにアクセシビリティを向上。}
        \resumeItem{新しいチームメンバーの指導を支援し、プロジェクトタスクの指導および面接に参加。}
      \resumeItemListEnd

  \resumeSubHeadingListEnd

%-----------PERSONAL PROJECTS-----------
\section{個人プロジェクト}
  \resumeSubHeadingListStart
    \resumeItemListStart
        \resumeItem{C++とDirectX12でゲームエンジンを開発。}
        \resumeItem{ソフトウェアラスタライザライブラリおよび小型仮想マシンをCで実装。}
        \resumeItem{TypeScript、llama-cpp、SQLite、およびORM Sequelizeを使用してAI LLMチャットボットを作成。}
        \resumeItem{Unityでカスタムスクリプタブルレンダーパイプラインおよびディファードレンダラーを実装。}
        \resumeItem{カスタムPBRおよびスクリーンスペースレイトレーシングシェーダーを開発。}
        \resumeItem{Unityでカスタムテレインシステムを作成。}
        \resumeItem{コンピュートシェーダーを使用してクラウドシステムアセットを作成。}
        \resumeItem{ラグドール物理学に取り組み、アクティブラグドールを実装し、レイキャストとホイール摩擦式を使用してカスタムカーフィジックスを開発。}
        \resumeItem{Blenderを使用したローポリモデリングとアニメーションに従事。}
    \resumeItemListEnd
  \resumeSubHeadingListEnd

%-----------EDUCATION-----------
\section{教育}
  \resumeSubHeadingListStart
    \resumeSubheading
      {ヴォルゴグラード州立大学}{ロシア、ヴォルゴグラード}
      {コンピュータサイエンス学士}{2016年9月 -- 2020年6月}
    \resumeSubheading
      {東京育英日本語学校}{日本、東京}
      {日本語学学生}{2023年7月 -- 現在}
  \resumeSubHeadingListEnd

%-----------TECHNICAL SKILLS-----------
\section{技術スキル}
 \begin{itemize}[leftmargin=0.15in, label={}]
    \small{\item{
     \textbf{スキル}{: C++, C\#, C, TypeScript, DirectX 12, CG/HLSL, ShaderLab, 数学, アルゴリズム.} \\
     \textbf{ツール}{: Unreal Engine, Unity3D, Perforce, Git, Blender, Visual Studio, Jira, TeamCity, Jenkins, RenderDoc.}
    }}
 \end{itemize}

%-------------------------------------------

\end{CJK*}
\end{document}
